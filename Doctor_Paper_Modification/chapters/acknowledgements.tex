% !TeX root = ../main.tex
\begin{acknowledgements}
不知不觉,论文已经写到尾声,临近毕业,心中有百感交集。读博的过程中有痛苦也有快乐,我永远也不会忘记那些一起爬装置搭光路的夜晚,一起走过的那段橘黄色路灯下的创新大道,和谢老师讨论问题的午后阳光,以及因为算法不收敛苦苦寻找几天几夜的程序bug。在离别之际,我想向一起陪伴过我、帮助过我的老师、同学致以真诚的感谢。

首先感谢我的导师刘万东老师给我宝贵的读博的机会,当初选择报考导师时就被刘老师的形象所吸引,觉得刘老师像文学家马尔克斯,事实上刘老师不仅物理功底深厚也确实热爱文学。刘老师非常爱护自己学生,平时聚会上刘老师也很愿意和我们学生分享他对社会和人生的思考。在大组中刘老师的形象就像是大家长,在这样的大家庭中,我们每一个学生才得以有了优越的科研环境来学习知识提高能力。虽然和刘老师接触不多,但刘老师的随和亲切的语言、博学的知识储备和低调的处事风格都使我印象深刻,非常荣幸能成为刘老师的学生。

我最要感谢的是谢锦林老师,本论文是在谢老师指导下完成的,研究生期间是谢老师指引着我在科学的海洋中探索未知的世界,给予我足够的信心和勇气去面对挑战。从微波成像实验到程序编写,从最简单的分束片选择到电子回旋辐射分析,研究生期间在谢老师指导下我收获了很多知识和技能。谢老师敏锐的物理直觉,抓住物理本质的本领和站在更高一层思考问题的能力让我受益良多,很多时候遇到问题找谢老师讨论我都能打开新的窗口看到不一样的风景。在看文献中谢老师要求我们要抱着批判的精神去阅读,不能人云亦云,要善于思考,有一次去吃饭的路上谢老师看到天边的白云突然问我们为什么白云是白色的,我们学的电动力学书只告诉我们天空为什么是蓝色可从来没告诉我们白云为什么是白色的,结果我们都没有回答出来,谢老师告诫我们学知识不能太死板,要能够触类旁通。谢老师常说对待学问要有钻研的精神,就算是扫地也能扫出文章,谢老师对待学问一丝不苟的态度,面对问题时冷静分析从容不迫的心态潜移默化地影响着我,让我懂得了如何去面对问题解决问题。时光流转,岁月如梭,毕业后虽然不在谢老师身边,但是谢老师教给我的人生态度和处事方法永远是我宝贵的精神财富,感谢您多年来给予我的悉心教导和帮助,学生朽木。
 
感谢庄革老师在科研上对我的点拨,庄老对方程理解的通透,对科学问题的深刻见解都让我十分钦佩。感谢丁卫星老师的给予我对科研工作的指导和帮
助,在实验和未来工作规划中丁老师都给过我宝贵建议;感谢刘健老师,刘健老师提供了无比珍贵的超算资源和优秀的学生协助
我解决计算问题,在和刘健老师的相处中我受益良多;感谢李弘老师精彩的等离子非线性理论课程,我学到了很多知识,李弘老师的儒雅博学给我留下了深刻的印象;感谢帅气的刘阿娣老师提供的实验帮助;感谢兰涛老师,兰老师写的EPPG程序让我学到了很多数据处理技巧;感谢周楚老师,Ray-tracing3D程序的开发离不开周楚老师曾经写的射线追迹代码,感谢毛文哲老师在实验上的帮助。感谢郑坚老师、马锦秀老师、杨维纮老师、丁卫星老师、朱晓东老师、王传兵老师开设的等离子相关课程,在这些课程中我不仅学到了渊博的等离子物理知识,各位老师严谨治学态度也是我学习的榜样。感谢等离子所老师们在实验上的关心和帮助,他们有姚远老师,刘海庆老师,提昂老师,曾龙老师,赵海林老师等,在此一并谢过。

感谢朱逸伦师兄在科研上的帮助,从小朱师兄这里我学到了很多关于光学设计和微波天线上有趣的知识;感谢杨尚川师兄,川哥扎实的理论基础曾帮我解
决了许多重要的数学问题;感谢高炳西师兄在实验上提供的帮助;感谢赵朕领师兄在学习上对我的指导,赵朕领师兄非常聪明又很谦和,在做人做事上赵
朕领师兄都树立了很好的榜样;我还要感谢陈东旭师兄在实验上和电路设计上对我的帮助还有廖望师兄在科研上给予我的支持;感谢邹志慧师兄和王彦鹏
师兄在学习上的讨论交流;在研究生生涯中,我特别要感谢渠承明师兄,是他带着我一起去做实验,一起分析问题解决问题。感谢高飞雪师妹和张立夫师
弟,在实验过程中是他们一直陪伴着我,无论刮风下雨,我们互相调侃,互相帮助,回想起来那是一段十分美好的时光;感谢李文祥师弟在数值模拟上给
予的热心帮助;感谢我们办公室的余基诺、裴柏杨、徐梦梦、林之意、刘悠然、曾庆彬、杨锦琛同学们的互帮互助;感谢李子涵、强子薇、张云娇、蔡宇诚、蒋一雄、潘迪、韩玉箫、张景硕、邵丹在读书报告会上和学习中的互相交流和帮助。

感谢与我同届的邬佳仁、李磐、左雨澍、季佳旭、叶凯萱,那些曾经一起学习的日子是一段宝贵的时光。感谢周呈熙同学和张琪同学的交流讨论。

感谢朱有娣秘书和陆玲玲秘书对我科研上的支持和帮助。


感谢我女朋友一直对我的支持,在生活中你和我分享你的开心,你的难过,你所鄙夷的小人,你也会抱怨我听不懂话,情商低,有时候我们还会因意见不和而争吵生气,但是在我腿抽筋时你会寄钙片给我,在我心情低落的时候你会陪伴着我,从相识到相知,我一直没有给你买过任何贵重的礼物送你,谢谢你的关心和陪伴,你就像彩虹一样出现在我的世界。

最后感谢我的家人,小时候我爸妈就告诉我要我努力读书,那时候我爸就常带我去集市书店买书,家里的童话书和探险书多到让周围的小伙伴们都很羡慕。因为家里经济拮据,好多玩具都是手工做的,我清晰的记得我爷爷用报纸和秸秆做了一个巨大的风筝,风筝后面的尾巴就用稻田里收割后的稻根拖着来维持受力平衡,还有我爸在削圆的白萝卜上画的世界地图(当时对世界地图并不感兴趣,只觉得把它当着球踢还挺好玩)、迎风旋转的大风车,这些都是我快乐源泉。作为从农村出来的人,读书时常听到有句话叫吃得苦中苦,方为人上人。后来才发现人上人是多么可笑,处在高处的和低处的人永远不平等,而低处的人又往往会因为利益纠纷勾心斗角,面对不公正的事时又无能为力,在这样的环境中我越发觉得努力读书应该是为了没有人上人。感谢父母一直以来对我的支持,祝愿你们平安幸福。

感谢伟大的祖国,愿人民幸福,祖国繁荣昌盛!

\hfill 2023年6月16日

\end{acknowledgements}

























