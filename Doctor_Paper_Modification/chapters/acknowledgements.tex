% !TeX root = ../main.tex
\begin{acknowledgements}
不知不觉间,论文已写到尾声,毕业在即,心中百感交集。读博的日子里有痛苦也有快乐,我永远不会忘记那些深夜里一起爬装置、搭光路的时光,橘黄色路灯下走过的创新大道,与谢老师讨论问题的午后阳光,以及为算法不收敛而苦苦排查数日的程序 bug。临别之际,我想向一路陪伴和帮助过我的老师、同学致以最真诚的感谢。

首先感谢我的导师刘万东老师,感谢您给予我宝贵的读博机会。当初选择报考时,我便被刘老师的形象所吸引,觉得刘老师颇有文学家马尔克斯的风采。事实证明,刘老师不仅物理功底深厚,也确实热爱文学。刘老师十分关心和爱护学生,在平日的聚会上,也乐于与我们分享他对社会和人生的思考。在大组中,刘老师的形象犹如大家长,在这样的集体中,我们每一位学生都得以拥有优越的科研环境来学习知识、提升能力。虽然与刘老师的直接接触不多,但刘老师随和亲切的谈吐、渊博的学识和低调的处事风格都令我印象深刻。我非常荣幸能够成为刘老师的学生。

我最要感谢的是谢锦林老师,本论文是在谢老师的指导下完成的。研究生期间,是谢老师指引我在科学的海洋中探索未知世界,给予我足够的信心和勇气去面对挑战。从微波成像实验到程序编写,从最基础的分束片选择到电子回旋辐射分析,在谢老师的悉心指导下,我收获了丰富的知识与技能。谢老师敏锐的物理直觉、把握物理本质的能力,以及高屋建瓴的思维方式,使我受益匪浅。每当遇到问题与谢老师讨论时,我总能开阔视野,获得新的启发。在阅读文献时,谢老师要求我们要抱着批判的精神去思考,不能人云亦云。有一次去吃饭的路上,谢老师看到天边的白云,突然问我们为什么白云是白色的。我们学的电动力学书只告诉我们天空为什么是蓝色,却从未告诉我们白云为什么是白色的,结果我们都答不上来。谢老师告诫我们,学知识不能太死板,要能够触类旁通。谢老师常说,对待学问要有钻研精神,就算是扫地也能扫出文章。谢老师对待学问一丝不苟的态度,以及面对问题时冷静分析、从容不迫的心态,潜移默化地影响着我,让我懂得了如何去面对和解决问题。时光流转,岁月如梭。毕业后,虽然不在谢老师身边,但谢老师教给我的人生态度和处事方法,将永远是我宝贵的精神财富。感谢您多年来给予我的悉心教导和帮助!学生朽木。
 
感谢庄革老师在科研工作中的悉心点拨,庄老师对方程的透彻理解和对科学问题的深刻见解令我十分钦佩。感谢丁卫星老师在科研工作中给予我的指导和帮助,在实验及未来工作规划方面为我提供了宝贵的建议。感谢刘健老师,刘老师不仅为我提供了极为珍贵的超算资源,还安排优秀的学生协助我解决计算问题,在与刘老师的交流中我受益良多。感谢李弘老师开设的精彩的等离子体非线性理论课程,使我收获颇丰,李老师的儒雅与博学给我留下了深刻的印象。感谢刘阿娣老师为我提供的实验帮助。感谢兰涛老师,其编写的 EPPG 程序让我学到了许多数据处理技巧。感谢周楚老师,Ray-tracing3D 程序的开发离不开周老师早期编写的射线追迹代码。感谢毛文哲老师在实验中的帮助。感谢郑坚老师、马锦秀老师、杨维纮老师、丁卫星老师、朱晓东老师、王传兵老师开设的等离子体相关课程,让我不仅学到了渊博的等离子体物理知识,更以各位老师严谨的治学态度为榜样。感谢等离子所的诸位老师在实验中的关心和帮助,包括姚远老师、刘海庆老师、提昂老师、曾龙老师、赵海林老师等,在此一并致谢。

感谢朱逸伦师兄在科研上的帮助,从小朱师兄那里我学到了许多关于光学设计和微波天线的有趣知识。感谢杨尚川师兄,川哥扎实的理论基础曾帮我解决了许多重要的数学问题。感谢高炳西师兄在实验中提供的帮助。感谢赵朕领师兄在学习上的指导,他既聪明又谦和,在做人做事方面都为我树立了很好的榜样。
我还要感谢陈东旭师兄在实验和电路设计上的帮助,以及廖望师兄在科研中给予我的支持。感谢邹志慧师兄和王彦鹏师兄在学习上的讨论与交流。在研究生生涯中,我尤其要感谢渠承明师兄,是他带着我一起做实验,一起分析问题、解决问题。
感谢高飞雪师妹和张立夫师弟,在实验过程中一直陪伴着我,无论刮风下雨,我们互相调侃、互相帮助,回想起来,那是一段十分美好的时光。感谢李文祥师弟在数值模拟上给予的热心帮助。
感谢我们办公室的余基诺、裴柏杨、徐梦梦、林之意、刘悠然、曾庆彬、杨锦琛同学的互帮互助。感谢李子涵、强子薇、张云娇、蔡宇诚、蒋一雄、潘迪、韩玉箫、张景硕、邵丹在读书报告会和学习中的交流与帮助。

感谢与我同届的邬佳仁、李磐、左雨澍、季佳旭、叶凯萱,那些曾经一起学习的日子是一段宝贵的时光。感谢周呈熙同学和张琪同学的交流讨论。

感谢朱有娣秘书和陆玲玲秘书对我科研上的支持和帮助。


感谢我的女朋友一直以来对我的支持。生活中,你与我分享你的开心、你的难过,以及你所鄙夷的小人;你也会抱怨我不够细心、情商不高。我们有时会因意见不合而争吵生气,但在我身体不适时,你总是关心我;在我心情低落时,你始终陪伴在我身边。从相识到相知,我从未送过你任何贵重的礼物,却始终感受到你真挚的关心与陪伴。你就像一道彩虹,照亮了我的世界。

最后,感谢我的家人。小时候,父母就告诉我要努力读书。那时,父亲常带我去集市的书店买书,家中的童话书和探险书多到让周围的小伙伴都羡慕。由于家境拮据,很多玩具都是手工制作的。我清晰地记得,爷爷曾用报纸和秸秆做了一个巨大的风筝,风筝的尾巴用麻绳拖着收割后的稻根来保持受力平衡;父亲还在削圆的白萝卜上画出世界地图(虽然当时对地图并不感兴趣,只觉得当作球踢很好玩);还有迎风旋转的大风车——这些都是我童年最纯粹的快乐源泉。
作为从农村走出来的人,读书时常听到一句话:“吃得苦中苦,方为人上人。”后来才慢慢意识到,“人上人”其实是多么可笑,身处高处的人和低处的人从未真正平等,而低处的人又常常因利益纠纷勾心斗角,面对不公时又无能为力。正因如此,我愈发觉得努力读书,不是为了成为“人上人”,而是为了让这个世界少一些人上人。
感谢父母一直以来的支持与付出,愿你们平安幸福。

感谢伟大的祖国,愿人民幸福,祖国繁荣昌盛!

\hfill 2023年6月16日

\end{acknowledgements}

























