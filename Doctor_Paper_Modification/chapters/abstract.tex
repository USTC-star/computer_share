% !TeX root = ../main.tex

\ustcsetup{
  keywords  = {电子回旋辐射;非热化电子;动理学;数值模拟;反常多普勒效应;逃逸电子},
  keywords* = {Electron Cycltron Emission,Non-thermal Electrons,Kinetic theory,Numerical simulation, Anomalous Doppler Effect,runaway electrons},
}

\begin{abstract}
 % 托卡马克放电过程中伴随大量的物理过程,这些过程互相耦合导致磁约束等离子体表现出繁杂丰富的实验现象。例如在EAST托卡马克电流爬升过程中,电子回旋辐射信号出现秒量级的前端峰结构并伴随着毫秒量级台阶结构,另一个现象是放电平台期密度下降到某个临界值时电子回旋辐射信号会突然开始暴增。除此之外,托卡马克在一定条件下会产生大量逃逸电子,高达MeV的高能逃逸电子不仅浪费欧姆储能,当其轰击在第一壁上还会对装置带来严重的破坏。随着放电参数的提高,逃逸电子问题对未来的聚变装置的影响也越发显著,但逃逸电子的动理学过程目前仍然有待进一步研究。逃逸电子的产生和电子回旋辐射的演化本质上都可以用分布函数来描述,而分布函数的演化又与动理学方程相关,为了理解反常电子回旋辐射现象的产生机制以及逃逸电子的动理学过程,本文通过对物理过程建模分析,利用数值驱动的方法根据诊断测量的背景参数驱动动理学模型运转,以此研究不同放电环境下电子回旋辐射的实验现象。\par
托卡马克中非热化电子直接与等离子体放电品质、装置安全运行等重要问题密切相关。研究非热电子的动理学机制及控制机理具有重要现实意义。非热电子在托卡马克放电过程中广泛存在,电子回旋辐射诊断对非热电子尤为敏感,在 EAST 托卡马克电流爬升过程中,非热化电子导致电子回旋辐射信号出现秒量级的前端峰结构和毫秒量级的台阶结构。在 EAST 托卡马克放电平台期,当密度下降到某个临界值时,电子回旋辐射信号也有时会突然暴增。本文基于实验测量的背景参数(如温度、密度、环电压等)通过动理学方程和电子回旋辐射数值诊断对以上实验现象展开了数值研究,以求探究电子回旋辐射实验数据背后更加深刻的物理过程。当托卡马克处于一定放电条件下时会产生大量逃逸电子,其中高达 MeV的高能逃逸电子不仅会消耗欧姆储能,还会对装置造成严重的破坏。随着放电参数的提高,逃逸电子问题对未来的聚变装置影响越来越大,亟需要有效可靠的手段抑制逃逸电子。因此本文对逃逸电子的控制问题展开了初步的探索。
\par
本文设计了从分布函数映射到回旋辐射强度的计算程序,为研究电子速度分布演化提供了可与实验参考的数值诊断平台。电子回旋辐射强度计算涉及了辐射率、吸收系数、辐射路径等问题,本文通过 B. Trubnikov 和 G. Bekefi 等科学家提出的理论完成了吸收系数和辐射率与分布函数关系的数值计算,结合辐射路径实现了电子回旋辐射强度和电子速度分布之间的定量计算。 为了计算电子速度分布的演化,本文考虑了均匀电场磁场(电场平行于磁场)背景条件下分布函数的动理学过程。动理学模型包含了电子碰撞项、辐射阻尼项、电子雪崩项、磁扰动扩散项等。最后通过标准模型和理论解对比,验证算法计算结果的准确性。 通过辐射输运计算程序和动理学程序,本文结合 EAST 实验中放电参数,对电子速度分布 演化和对应的回旋辐射进行求解,揭示了电子回旋辐射前端峰形成过程中逃逸电子雪崩机制参与的重要作用。针对放电平台期密度下降到某个临界值时,电子回旋辐射信号也有时会突然暴增的实验现象,本文确认了EAST托卡马克中密度降低时电子回旋辐射信号迅速上升的根本原因是磁扰动引起的非热电子损失项减少。通过数值模拟和实验数据进一步验证了这个结论。 最后本文研究反常多普勒效应的动力学过程,从单粒子运动出发,模拟了均匀背景下磁化电子和电磁波的相互作用过程。数值结果表明当电磁波电场强度超过背景静电场强度约 5倍,电子速度满足反常多普勒共振条件时,外界电磁波可将电子平行方向电场能量散射到垂直方向。同时电子还会激发出和背景电磁波相同性质的电磁波,电子平行方向速度此时不再被静电场加速。结合托卡马克装置中电磁波色散关系,对利用电磁波抑制逃逸电子方案的可行性进行了验证。该发现为磁约束装置运行中慢化逃逸电子提供了一种可能的方案。
\par
本文主要内容是独立开发完成了电子动理学数值计算程序及电子回旋辐射诊断数字仿真平台,并利用标准模型验证了程序的正确性。然后基于已开发的数值模拟程序,确认了EAST托卡马克中密度降低时电子回旋辐射信号迅速上升的根本原因是磁扰动引起的非热电子损失项减少;最后是基于保体积算法开发了单电子与外加电磁场相互作用的数值模拟程序。主要创新点是发现一定强度的电磁波可以通过反常多普勒共振效应抑制电子在平行磁场方向的加速,并提出了一种利用较低能量电磁波从托卡马克高场侧注入以控制托卡马克中逃逸电子能量的方法。

\end{abstract}

\begin{abstract*}
\par
Non-thermal electrons in tokamaks are directly associated with important issues such as plasma discharge quality and device safety. Investigating the kinetic mechanisms and control mechanisms of non-thermal electrons holds significant practical significance. Non-thermal electrons are widely present in tokamak discharge processes, and electron cyclotron emission diagnostics are particularly sensitive to non-thermal electrons. In the EAST tokamak, during the current ramp-up phase, the presence of non-thermalized electrons leads to the appearance of front-end peaks and millisecond-level step structures in the electron cyclotron emission signal. Additionally, during the discharge plateau phase in the EAST tokamak, there are instances where the electron cyclotron emission signal suddenly increases when the density drops to a critical value.
In this study, numerical investigations were conducted on the aforementioned experimental phenomena using background parameters measured during experiments, such as temperature, density, and loop voltage. By employing kinetic equations and numerical diagnostics of electron cyclotron emission, a deeper understanding of the underlying physical processes behind the experimental data was sought. When a tokamak operates under certain discharge conditions, a large number of runaway electrons are generated. Among them, high-energy runaway electrons reaching MeV levels not only consume Ohmic energy but also cause severe damage to the device. As discharge parameters increase, the impact of runaway electrons on future fusion devices becomes increasingly significant, necessitating effective and reliable means to suppress runaway electrons. Hence, this study conducted preliminary explorations on the control of runaway electrons.
\par
A calculation program was designed in this study to map the electron velocity distribution to the cyclotron radiation intensity, providing a numerical diagnostic platform that can be referenced against experiments to study the evolution of electron velocity distribution. The calculation of electron cyclotron radiation intensity involves issues such as radiation rate, absorption coefficient, and radiation path. In this study, numerical calculations of the absorption coefficient and radiation rate in relation to the distribution function were performed based on theories proposed by B. Trubnikov and G. Bekefi, among other scientists. By combining the radiation path, a quantitative calculation of the electron cyclotron radiation intensity and electron velocity distribution was achieved. To calculate the evolution of the electron velocity distribution, the study considered the kinetic processes of the distribution function under the background conditions of a uniform electric field and magnetic field (with the electric field parallel to the magnetic field). The kinetic model included electron collision terms, radiation damping terms, electron avalanche terms, and magnetic perturbation diffusion terms. The accuracy of the algorithm's calculation results was verified by comparing them with those of standard models and theoretical solutions. By utilizing radiation transport calculation programs and kinetic programs, this study combined discharge parameters from EAST experiments to solve the evolution of electron velocity distribution and its corresponding cyclotron radiation. The significant role played by the runaway electron avalanche mechanism in the formation of the front-end peak in electron cyclotron radiation was revealed. Regarding the experimental phenomenon where the electron cyclotron emission signal suddenly increases when the density drops to a critical value during the discharge plateau phase in the EAST tokamak, this study confirmed that the fundamental reason for the rapid rise in the electron cyclotron emission signal is the reduction of non-thermal electron loss caused by magnetic perturbations. This conclusion was further validated through numerical simulations and experimental data. Finally, this study investigated the dynamic process of anomalous Doppler effect and simulated the interaction between magnetized electrons and electromagnetic waves starting from the motion of individual particles under a uniform background. Numerical results indicate that when the electric field intensity of the electromagnetic wave exceeds approximately five times the intensity of the background electrostatic field and the electron velocity satisfies the condition for anomalous Doppler resonance, the external electromagnetic wave can scatter the energy of the electron's parallel electric field to the perpendicular direction. At the same time, electrons also excite electromagnetic waves with the same properties as the background electromagnetic waves, and the parallel velocity of the electrons is no longer accelerated by the electrostatic field. Combined with the dispersion relationship of electromagnetic waves in tokamak devices, the feasibility of using electromagnetic waves to suppress runaway electrons was verified. This discovery provides a potential approach for slowing down runaway electrons in magnetic confinement devices.
\par
The main content of this study involves the independent development of numerical calculation programs for electron kinetics and a digital simulation platform for electron cyclotron radiation diagnostics. The correctness of the programs was verified using standard models. Based on the developed numerical simulation program, the study confirmed that the fundamental reason for the rapid rise in the electron cyclotron emission signal when the density decreases in the EAST tokamak is the reduction of non-thermal electron loss caused by magnetic perturbations. Finally, a numerical simulation program was developed based on the conservation volume algorithm to simulate the interaction between individual electrons and external electromagnetic fields. The main innovation is the discovery that certain intensity of electromagnetic waves can suppress the acceleration of electrons in the direction of parallel magnetic field through the anomalous Doppler resonance effect. It also proposes a method to control the energy of runaway electrons in a tokamak by injecting lower energy electromagnetic waves from the high-field side.
\end{abstract*}














