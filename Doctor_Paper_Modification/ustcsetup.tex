% !TeX root = ./main.tex

\ustcsetup{
  title              = {托卡马克中非热化电子动理学演化及其对回旋辐射影响的数值研究},
  title*             = {Numerical study of the kinetic evolution of non-thermal electrons in Tokamak and its influence on cyclotron radiation},
%  title              = {****},
%  title*             = {****},
  author             = {徐新航},
  author*            = {Xu Xinhang},
%  author             = {***},
%  author*            = {***},
  speciality         = {等离子体物理},
  speciality*        = {Plasma Physics},
%  speciality         = {***},
%  speciality*        = {***},
  supervisor         = {刘万东~教授, 谢锦林~教授},
  supervisor*        = {Prof. Liu Wandong, Prof. Xie jinlin},
%  supervisor         = {***, ***},
%  supervisor*        = {***, ***},
   date               = {2023-09-10},  % 默认为今日
  % professional-type  = {专业学位类型},
  % professional-type* = {Professional degree type},
  % department         = {数学科学学院},  % 院系,本科生需要填写
  % student-id         = {PB11001000},  % 学号,本科生需要填写
  % secret-level       = {秘密},     % 绝密|机密|秘密|控阅,注释本行则公开
  % secret-level*      = {Secret},  % Top secret | Highly secret | Secret
  % secret-year        = {10},      % 保密/控阅期限
  % reviewer           = true,      % 声明页显示“评审专家签名”
  %
  % 数学字体
  % math-style         = GB,  % 可选:GB, TeX, ISO
  math-font          = xits,  % 可选:stix, xits, libertinus
}


% 加载宏包

% 定理类环境宏包
\usepackage{amsthm}

% 插图
\usepackage{graphicx}

% 三线表
\usepackage{booktabs}

% 跨页表格
\usepackage{longtable}

% 算法
\usepackage[ruled,linesnumbered]{algorithm2e}

% SI 量和单位
\usepackage{siunitx}

% 参考文献使用 BibTeX + natbib 宏包
% 顺序编码制
\usepackage[sort]{natbib}
\bibliographystyle{ustcthesis-numerical}

% 著者-出版年制
% \usepackage{natbib}
% \bibliographystyle{ustcthesis-authoryear}

% 本科生参考文献的著录格式
% \usepackage[sort]{natbib}
% \bibliographystyle{ustcthesis-bachelor}

% 参考文献使用 BibLaTeX 宏包
% \usepackage[style=ustcthesis-numeric]{biblatex}
% \usepackage[bibstyle=ustcthesis-numeric,citestyle=ustcthesis-inline,url=false]{biblatex}
% \usepackage[style=ustcthesis-authoryear]{biblatex}
% \usepackage[style=ustcthesis-bachelor]{biblatex}
% 声明 BibLaTeX 的数据库
% \addbibresource{bib/ustc.bib}

% 配置图片的默认目录
% 加载宏包
\usepackage{graphicx}
\usepackage{epstopdf}
\usepackage{booktabs}
\usepackage{longtable}
\usepackage[ruled,linesnumbered]{algorithm2e}
\usepackage{siunitx}
\usepackage{amsthm}
\usepackage{cancel}
\usepackage{xcolor}
\usepackage{appendix}
\usepackage{verbatim}  % comment multi lines

\usepackage{float}
\setmainfont{Times New Roman} 
\usepackage{overpic}
\usepackage{multirow}
\usepackage{longtable}
\usepackage{makecell}
\usepackage{diagbox}
\input{math-commands.tex}
\graphicspath{{figures/}}
\ustcsetup{
  cite-style = super,
}
% 数学命令
\makeatletter
\newcommand\dif{%  % 微分符号
  \mathop{}\!%
  \ifustc@math@style@TeX
    d%
  \else
    \mathrm{d}%
  \fi
}
\makeatother
\newcommand\eu{{\symup{e}}}
\newcommand\iu{{\symup{i}}}

% 用于写文档的命令
\DeclareRobustCommand\cs[1]{\texttt{\char`\\#1}}
\DeclareRobustCommand\env[1]{\texttt{#1}}
\DeclareRobustCommand\pkg[1]{\textsf{#1}}
\DeclareRobustCommand\file[1]{\nolinkurl{#1}}

% hyperref 宏包在最后调用
\usepackage{hyperref}
